\documentclass[17pt]{extarticle}
\usepackage{graphicx} 
\usepackage[english, russian]{babel}
\usepackage{amsmath}
\usepackage{geometry}
\usepackage{graphicx}
\usepackage{makecell}

\geometry{
    a4paper,
    margin=0.5in,
}

\begin{document}
\pagestyle{empty}
\begin{flushleft}
УДК 519.715
\end{flushleft}

\noindent ПОСТРОЕНИЕ ГРАФА ПЕРЕХОДОВ ПОСЛЕДОВАТЕЛЬНОСТНОЙ \\ 

\vspace{-2.5em}
\begin{center}
СХЕМА
\end{center}

\vspace{-2.5em}
\begin{center}
С ПРИМЕНЕНИЕМ SAT-РЕШАТЕЛЯ
\end{center}

\vspace{-2em}
\begin{center}
Иванов Иван Иванович \\
кандидат технических наук, доцент\\
Россия, Томск \\
Петров Петр Петрович \\
студент,Томский государственный университет \\ 
Россия, Томск
\end{center}

Аннотация. Рассматривается подход к построению графа переходов последовательностной схемы. Исследуется метод, основанный на использовании SAT-решателя. Для построения графа переходов определяются возможные переходы между состояниями последовательной схемы и используются предварительные вычисления, основанные на троичном и двоичном моделировании, значительно сокращающие объем вычислений. Также рассматривается построение на основе графа переходов последовательностей, обеспечивающих заданные переходы схемы. Компьютерные эксперименты показали эффективность предложенного метода построения графа переходов с применением SAT-решателя. \\
\indentКлючевые слова: граф переходов, троичное моделирование, SAT-решатель, последовательностная схема, переходная последовательность.

В работе рассматривается подход к построению графа переходов синхронной последовательностной схемы. Построение графа, основанное на использовании SAT-решателя, исследуется подробно. Представлены результаты компьютерных экспериментов для предложенного метода построения графа переходов, применяющего SAT-решатель. Также кратко рассматривается решение задачи построения последовательности входных векторов, обеспечивающей переход схемы в одно из состояний заданного множества, по графу переходов.

\newpage

Рассмотрим синхронную последовательностную схему с $n$ входами, $m$ выходами и $p$ элементами памяти (триггерами). \\ $X=\left\{x_1,...,x_n\right\}$ - входные переменные схемы, $Y=\left\{y_1,...,y_m\right\}$ - ее выходные переменные, $Z=\left\{x_1,...,x_n\right\}$ – внутренние переменные схемы.

Назовем \textit{графом переходов последовательностной схемы} ориентированный граф, у которого вершины сопоставлены состояниям схе-мы и есть дуга из вершины $i$ в вершину $j$ тогда и только тогда, когда в схеме существует одношаговый переход из состояния, соответствую-щего вершине $i$, в состояние, соответствующее вершине $j$, при каких-либо значениях входных переменных.
\\
\indent На рисунке 1 представлена комбинационная составляющая С по-следовательностной схемы. При построении графа переходов рассмат-ривается только та часть схемы, которая необходима для получения функций переходов, выходы схемы не рассматриваются. Поэтому структурное описание комбинационной составляющей, используемой для получения графа переходов, упрощается (рисунок 2). 


\begin{figure}[h]
    \begin{minipage}{0.45\textwidth} 
        \includegraphics[width=\textwidth]{1pic.png} 
        \caption{Ирис \textit{setosa}}
        \label{fig:left}
    \end{minipage}
    \hfill 
    \begin{minipage}{0.45\textwidth} 
        \includegraphics[width=\textwidth]{3pic.png} 
        \raggedleft
        \caption{Ирис \textit{virginica}}
        \label{fig:right}
    \end{minipage}
\end{figure}

В схеме с рисунка 2 можно исключить все элементы, не связанные с ее выходами, то есть с входами триггеров последовательностной схемы.
Система функций переходов последовательностной схемы имеет вид:
\begin{equation}
    z_j^t = \Psi_j(x_1^t,...,x_n^t,z_1^{t-1},...,z_p^{t-1}), j=\overline{1,p}.
\end{equation}
Будем представлять полное состояние схемы вектором ($a,\delta$), где $a$ - вектор значений входных переменных X, а $\delta$ вектор значений внутренних переменных Z.

Двоичный вектор $\tau^i = (\tau_1^i,...,\tau_p^i)$ значений переменных Z будем называть \textit{кодом состояния} $q_i$. Q = ${q_1,...,q_t}$, где $t = 2^p$, – множество всех состояний схемы.  \\
\indent Рассмотрим общий подход к построению графа переходов после-довательностной схемы предложенный в работах [1, 2] и других. \\
\indent Представленные свойства определены в [3]. \\
\indent Рассмотрим подробнее следующее свойство, сформулированное в [3], используемое на 2-ом шаге сокращения вычислений. \\
\indent \textit{Пусть выполнено точное троичное моделирование функций пе-реходов системы (1) на векторе $(a,\delta)$, представляющем полное состояние, и получен вектор значений внутренних переменных $\delta'$. $\delta'$ представляет минимальный покрывающий интервал множества булевых векторов значений переменных Z, а не точное множество этих век-торов. Таким образом, множество состояний схемы достижимых за один шаг из множества $N(a,\delta)$ может быть подмножеством множества N$(\delta').$} \\  
\indent Результаты экспериментов \\
\indent Для проверки эффективности предложенного метода построения графа переходов последовательностной схемы, использующего SAT-решатель, были проведены эксперименты на бенчмарках. Эксперимен-ты проводились на контрольных примерах (бенчмарках) ISCAS’89, представляющих последовательностные схемы. Для оценивания рабо-ты исследуемого метода при проведении экспериментов для различ-ных бенчмарок измерялось время построения графа переходов, а также процент определяемых значений элементов матрицы $M$ на каждом ша-ге предварительных вычислений. Результаты компьютерных экспери-ментов представлены в таблице 1. 

\newpage 

\begin{center}
    Таблица 1 – Результаты построения графов переходов

\end{center}

\begin{table}[h]
    \centering
    \small
    \begin{tabular}{|p{0.9cm}|p{0.9cm}|p{0.9cm}|p{1.6cm}|p{1.5cm}|p{1.7cm}|p{2.5cm}|p{3cm}|p{2.5cm}|} 
        \hline
        \makecell{Бенч\\-\\мар-\\ки} & \makecell{Вхо-\\ды\\ (\#)} &  \makecell{Вых-\\оды\\ (\#)} & \makecell{Элемен-\\ты\\ памяти \\(\#)} & \makecell{Элемен\\ты \\(\#)} & \makecell{Среднее\\время\\ постро-\\ения\\ граф\\(сек.)} & \makecell{Определен\\ные\\ переходы на\\ шаге 1\\ (только 1) \\(\%)} & \makecell{определен\\ные\\ переходы на\\ шаге 2\\ (только 0)\\ (\%)} & \makecell{соотноше\\ние 0 и 1 в\\ матрице М\\ (#0; #1)} \\ 
     \hline
        S27 & 4 & 1 & 3 & 10 & 0,01 & 17,19 & 31,25 & 31; 33 \\ 
        \hline
        S386 & 7 & 7 & 6 & 159 & 1,01 & 2,15 & 92,77 & 3800; 296 \\ 
        \hline
        S832 & 18 & 19 & 5 & 287 & 7,60 & 11,13 & 61,23 & 710; 314 \\ 
        \hline
        S510 & 19 & 7 & 6 & 211 & 21,84 & 2,46 & 97,36 & 3995; 101 \\ 
        \hline
        S1488 & 8 & 19 & 6 & 653 & 4,82 & 3,13 & 82,25 & 3369; 727 \\ 
        \hline
    \end{tabular}
    \label{tab:example} 
\end{table}

В таблице представлены следующие данные: имя бенчмарки, количество входов, выходов, элементов памяти и элементов схемы; среднее время построения графа переходов схемы (по трем экспериментам); процент существующих одношаговых переходов в графе, определенных на шаге 1 предварительных вычислений, процент не существующих одношаговых переходов, определенных на шаге 2 предварительных вычислений, и соотношение 0 и 1 в полученной матрице M. \\
\indent Шаги предварительных вычислений в экспериментах для рассмотренных бенчмарок определили от 48,44\% до 99,82\% одношаговых переходов (существующих и не существующих в графе). Шаг 2 предварительных вычислений, выполненный с помощью троичного моделирования, позволил определить большую часть одношаговых переходов, отсутствующих в графе. Значительная часть всех одношаговых переходов графа вычислена с помощью шагов 1 и 2 предварительных вычислений. Графы переходов для схем среднего размера построены за несколько секунд.

\newpage
\begin{center}
    Список литературы  
\end{center}

1. Иванов И.И. Построение графа переходов последовательностной схемы // Современные проблемы физико-математических наук: материалы IV Всероссийской науч.-практ. конф. с международным участием, (22 – 25 ноября 2018 г., г. Орёл). – Орел: ОГУ им. И.С. Тургенева, 2018. – Ч. 1. – С. 230 – 235. \\
2. Ivanov I. Three-Value Simulation of Combinational and Sequential Circuits and its Applications // 2020 IEEE Moscow Workshop on Electronic and Networking Technologies (MWENT). Moscow, Russia. 11−13 March 2020. – 7 pp. \\
3. Иванов И.И. Интервальные расширения булевых функций и троичное моделирование последовательностных схем // Таврический научный обозреватель. – 2017. – №5 (22). – С. 208(220. \\
4. Иванов И.И. Троичное моделирование комбинационных и синхронных последовательностных схем // Современные проблемы физико-математических наук / материалы V Всероссийской науч.-практ. конф. с международным участием, (26 – 29 сентября 2019 г., г. Орёл). – Орел: ОГУ им. И.С. Тургенева, 2019. – С. 274 – 281.

\end{document}
